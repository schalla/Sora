\documentclass[pdftex,12pt]{article}
\usepackage{mysty}
\begin{document}

\section*{Executive Summary}
\label{sec:exec}

Biofuels present a number of advantages in social, political, and
environmental regards, including decreased dependency on foreign oil,
employment opportunities to support the new infrastructure, and
mitigation of pollutants driving climate change. The Obama
Administration plans to produce 36 billion gallons of biofuels
annually by 2022, of which only 15 billion can come from the current
leading feedstock of corn. With the limited availability of corn for
biofuel production, a next generation biofuel feedstock must be
developed.

Algae has shown promise at the lab level, having the highest lipid
concentration of any feedstock. As a monoculture, it can further be
optimized through species selection and bioengineering. This leads to
significantly higher land efficiency, which is central to the design
of Sora's algae farm bioreactor. This bioreactor's novel design
incorporates the controllability of closed systems while reducing
structural support needs and concurrently increasing the effective
light penetration depth. Furthermore, by obtaining the carbon used for
growing algae from coal-fired power plants and other factories, this
system can assist in sequestering carbon emissions.

There are several algae farms already in the market. However they
suffer from two main issues, namely in the lack of scalability and the
open pond design. There are also a handful of smaller operations
producing algae for producing small batches of biofuels. These
companies are splitting their efforts between growing algae and
optimizing the processing. Sora's aim is to specialize in algae growth
and potentially work with these companies for comparative advantage,
ultimately feeding into the Obama Administration's vision. Thus,
competitors become a major market once production increases are in
demand. In the meantime, Sora's algae production in its early stages
may feed into existing lucrative industries, including pharmaceuticals
and nutritional supplements.

Once the lab phase is completed, the pre-pilot phase begins. This
phase will allow for a demonstration of some of the advantages the
Sora design promises, and the output although in smaller quantities
can already be sold to customers. The pilot phase will follow, which
is a larger iteration of the pre-pilot with any necessary
adjustments. This and the following demo phase will serve to prove the
scalability of the design, and further customers and partners will be
identified. Appreciating the difficulty of expanding a new industry so
quickly, the target date for commercialization is between 2015 to
2020, in time to meet the 2022 deadline for increased biofuel
production. Once commercialization begins, the current financial
analysis indicates that break-even and ensuing profit will occur after
four years.

Biofuels, particularly those derived from algae, provide the solution
to many of today's problems. The new industry will create enormous
opportunities for employment while decreasing foreign dependency. The
process consumes carbon, holding off global climate change. By serving
as a "green fuel," increased production will also decrease mining and
drilling operations which have a demonstrated history of
harm. By serving to clean up the world for four years, Sora will break
even, after which profits will allow for further improvements to the
technology and thus to the world.

\end{document}
