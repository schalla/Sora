\documentclass[pdftex,12pt]{article}
\usepackage{mysty}
\begin{document}

\section*{Executive Summary}
\label{sec:exec}
Hi kento, this will be slightly more difficult
Biofuels will usher social, political, and environmental developments,
like decreased dependency on foreign oil, mitigation of pollutants,
and slowed climate change. The Obama Administration \textit{Growing
  America's Fuels} plan hopes to produce 36 billion gallons of biofuels
annually before 2022 and is limiting the use of currently popular corn
feedstocks to just 15 billion gal/year. The remaining 21 billion
gal/year must come from new, advanced feedstocks.

In labs algae outshines all other feedstocks, having the highest
useable energy concentration. As a monoculture, it can be optimized
through species selection and bioengineering leading to greater land
efficiency. Sora hopes to build the farms to take advantage of algae's
potential. The bioreactor's novel design effectively uses control
afforded by closed systems to increase the effective light penetration
depth and maximize monocultured growth efficiency. Better, by
obtaining the \co\ that feeds the  farms from factories and power
plants, this system can sequester carbon emissions on the side.

Today, there are already a few algae farms. However most suffer from
two main issues: a lack of scalability and open-pond designs. The rest
are directly associated with biofuel companies and are splitting their
efforts between growing algae and optimizing fuel processing. Sora's
aim is to specialize in algae growth and work with these companies in
order to achieve comparative advantage, accelerating the hopes of the
the political vision. Sora's competitors thus become its major market,
asking to supply their exponential demand. Even today the world needs
algae --- lucrative industries like pharmaceuticals and nutritional
supplements use mass quantities --- and Sora hopes to provide for
today and the future.

Sora's initial steps involve lab and pre-pilot phases to demonstrate
and hone its design while already producing salable biomass. The pilot
and demo phases follow as larger and larger iterations including newly
learned lessons and demonstrate the baked-in scalability of the
design. From these, we hope to attract further customers and strategic
partners. Appreciating the difficulty of expanding in a capital
intense industry, the target date for commercialization is between
2015 to 2020, in time to meet the 2022 deadline for increased biofuel
production. Once commercialization begins, the current financial
analysis indicates that break-even and ensuing profit will occur after
four years.

Biofuels, particularly those derived from algae, provide the solution
to many of today's problems. This new industry will create enormous
opportunities for employment and decrease foreign oil dependency. With
algae the process can even consume waste carbon, staving off global
climate change. Such green fuels help the environment systemically,
even reducing our own harmful drilling operations. By striving to help
our society and clean up the world, Sora hopes to provide a valuable
service and, by using efficient technology, will make it profitable.

\end{document}
