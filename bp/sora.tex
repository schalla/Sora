\documentclass[pdftex,12pt]{article}
\usepackage{mysty}
\begin{document}

\maketitlepage

\setcounter{tocdepth}{3}
\tableofcontents
\newpage

\section*{Executive Summary}
\label{sec:exec}
The United States is in the process of developing and scaling its
fledgling biofuels industry. Furthermore, the Obama Administration is
keen on pursuing biofuels production on an increasingly national
scale. Research indicates that the best feedstock for biofuels is
algae biomass. Consequently, there exists a growing demand from the
biofuels industry for large quantities of pure, cheap algae
biomass. In addition to the rapidly growing biofuels industry, algae
biomass is also a useful raw material in the pharmaceutical, food,
cosmetic, and fertilizer industries. It is Sora's goal to provide
large volumes of cheap, pure algae biomass to enable the continued
growth and development of dependent industries. 

\subsection*{Technology Overview}
\label{sec:exec-tech}

Algae is a robust photosynthetic microorganism known for its rampant
``algal blooms'' where, when not limited by resource scarcity, small
amounts of algae --- one of the fastest growing organisims on the
planet --- can rapidly grow to dominate an environment. Since this
process involves the consumption of carbon dioxide and its fixture
into algae biomass, Sora seeks to install effective algae bioreactors
on location at power plants in order to mitigate its emission into the
air. These bioreactors serve support the algae in a high-growth
environment enabling rapid carbon fixture and then serve to harvest
the algae creating dry algae biomass which is becoming widely salable.

\subsection*{The Market Opportunities}
\label{sec:exec-market}

Sora is positioned to provide value to two markets. Our primary
customers will be the power companies responsible for running
\co-emitting power plants, where Sora operates as a method to
aleviate environmental, fisal, and political pressure as a ``clean''
carbon method. Our secondary market is the sale of the algal biomass
which can be used in growing markets from advanced, custom biofuels to
cosmetics and food. As many of these sale markets are still in their
infancy, it is expected that we will be able to enter as a major
supplier in step with their increasing needs for algae.

Specifically, Sora seeks to work closely with individual power plants
to install our flexible bioreactor technologies on-site. Here we can
benefit from the power plant's existing support infrastructure and
reap benefits in not needing complex transport solutions. Moreover,
our bioreactors are designed to scale enabling power plants to select
installations that meet their needs precisely and that will grow as
those needs grow.

\subsection*{Competition}
\label{sec:exec-competition}

The problems in the energy sector are anything but unknown, however we
offer strong advantages over our primary competitors, Geologic Carbon
Capture and Sequestration (GCCS). The primary goal of GCCS is to
liquefy \co\ under extreme pressure and then inject the \co\ in this
unstable form into previously drilled oil wells. Not only does this
method involve intense energy expenditure on part of the plants but it
also requires careful monitoring of the storage sites deep into the
future, possibly even beyond the life of the GCCS companies
themselves. Algae is natural, solar-driven, and does not create
environmental time bombs as a side-effect. That being said, other
algae-based competitors exist as well, but our increased focus on
coal-burning power plants as a customer allows us to produce
technologies that are both cheaper and exhibit a closer synergy with
power plant facilities.

Within our secondary market we will also face competition in the
production of salable algae. Here we expect to dominate due to the
pairing of algae growth and carbon sequestration allowing us to turn
waste into value and operate with profit on both sides. Other algae
suppliers will have to absorb the expenses of their facilities and
\co\ directly.

\subsection*{Growth Profile}
\label{sec:exec-growth}

Sora's long-term objectives are today broken down into a six month
plan and a five year plan. By May 2010 we plan to develop Sora to the
point of having developed a pre-pilot installation to demonstrate and
test our technology. Additionally, we expect to have a full patent
application covering our core technology completed before January
2010. During this time we also plan to solidify the managerial side of
Sora by completing formal arrangements with our Board of Advisors and
recruiting a CEO with experience in startups, the energy sector, and
deal negotiation.

Our five year goals will eclipse the creation and sale of a full
commercial installation at a power plant. We'll achieve this by a
stepped progression of technology implementations from our 6-month
pre-pilot through a pilot installation at a partnered power plant or
industrial plant consisting of only one or two bioreactor units by the
second year. Finally, by the 4th year we expect a demonstration
low-volume installation at a power plant which will be the final
marketable proof-of-concept before the commercial phase.

\subsection*{The Bottom Line}
\label{sec:exec-bottomline}

Sora expects that within X years we will have full-scale installations
at some of the largest, most polluting power plants in the
U.S. enabling them to make their carbon quotas. Each installation is
expected to produce \$X/year on average and will cost us \$X/year to run
leading to \$X/year net profit. Additionally, the value of the produced
algae is expected to be at least \$X/ton enabling each installation to
produce \$X/year in secondary profit through sales. Finally, a world
with Sora bioreactors installed at power plants is both immediately
more environmentally sustainable but also poised for a future
supportive of sustainable energy.

\newpage
\section{Products and Services}
\label{sec:products_services}

\subsection*{Product}

\subsubsection*{Description and Function}
\label{product:description}

Sora's product is a system designed to generate pure algae biomass on
a large scale and for a low price. The product also has the capacity
to sequester moderate amounts of \co\ from any \co\ source. This
system is composed of two major subsystems: the algae growth region
and the stream processing units.

[Some details of technology removed]

\begin{comment}
  The algae growth region consists of a lattice of interwoven,
  transparent polymer tubes exposed to sunlight and containing an
  algae and water solution flowing through the tubes. There are many
  possible variations of the weave that could be implemented, based on
  the conditions of the sequestration site and including factors such
  as terrain slope and sunlight intensity. For instance, several tubes
  might be twisted or braided together, then further woven into a
  textile composed of the tubes.

  The algae solution flowing through the growth region passes through
  the stream processing units at regular intervals. These processing
  units are responsible for introduction and dissolution of nutrients
  and \co\ into the algae solution, for solution cooling, and for
  removal of oxygen and a small portion of the solution itself. Though
  it is possible that these elements may be incorporated into a single
  tower, the design of the stream processing unit currently consists
  of several components in series.

  Carbon dioxide will be dissolved into the algae solution using
  countercurrent flow, with flue gases bubbling upwards and water
  flowing downwards, allowing for the most complete transfer of the
  gases. Dissolved oxygen within the algae solution, beyond a certain
  limit, will inhibit the uptake of \co\ by the algae. Excess
  dissolved oxygen will be removed from the solution using sparging, a
  technique in which a gas passed through a liquid may be used to pull
  another gas out of solution. The portion of the algae solution that
  is harvested will be refined using a combination of flocculation,
  filtration, and pressing to obtain a more concentrated algae
  product.
\end{comment}

\subsubsection*{Key Features and Needs Satisfied}
\label{product:features}
As the biofuels industry continues to expand, there will be an ever
increasing demand for cheap and pure algae biomass. Currently, those
in the process of commercializing biofuels are more focused on the
process of extracting usable hydrocarbons from algae than on producing
cheap, pure algae on a large scale. As a result, current biofuels
projects are increasingly unable to economically scale their algal
production while maintaining purity to meet the needs of their growing
biofuels extraction processes. In this market, Sora's product
fullfills the need for an ever growing demand for large volumes of
pure, cheap algae biomass. Additionally, the pure cheap algae produced
by our system also serves to satisfy the needs of other consumers of
algae biomass, such as the pharmeceutical, food, cosmetic, and
fertilizer industries. 

In addition to providing a large stock of algae biomass, the product
also has the capacity to sink large volumes of \co\. Thus, the product
also offers heavy carbon dioxide polluters an effective, reliable, and
sustainable method to sequester their \co output.

\subsubsection*{Advantage of the Product}
\label{product:advantage}

While there exist many algae biomass producers, few are focused solely
on the large scale production of algae biomass. Essentially all
producers are also vested in the biofuels extraction process, and are
thus forced to split resources between algae production and biofuels
extraction. Sora, by focusing solely on the large scale production of
pure algae biomass, is well equipped to outcompete those forced to
split their resouces.

\paragraph{Compared to Other Closed Bioreactor Systems}
\label{product:compare_closedloop}

Increasingly, those in the field of large scale algae biomass
production are looking to closed systems. Sora's edge in this regard
lies in the utilization of a novel tubing system which requires far
less external structural support than any other existing large scale
closed-loop algae bioreactors. Our product is also designed to take
better advantage of the light incident upon a given area of land by
increasing the effective penetration depth. This feature is achieved
again through our novel physical layout. The combination of better
land use and cheaper support infrastructure impart a key advantage to
Sora's algal biomass production system over other closed loop systems.

\paragraph{Compared to Open Pond Systems}
\label{product:compare_openpond}

An alternative to closed loop systems are the open pond systems, an
older technology, in which algae is grown in ponds exposed to open
air. One of the benefits of our system over open pond systems is the
additional control we are able to exert over the flow of water. By
passing the algae solution through our bioreactor system, we are able
to control the average exposure of algae to light, as well as the
frequency through which algae pass through a light-dark cycle. Open
pond systems are only able to approximate this through controlled
mixing, leaving many parameters undefined and difficult to
quantify. Additionally, our system protects against contamination from
other algal species, while open pond systems are open to invasive
algal species competing with the desired algae, resulting in lower
growth efficiencies. The combination of various elements targeted at
improving our system's efficiencies over other systems, including open
ponds, will allow us to optimize the most important metric: our
system's bioproductivity per dollar.

\paragraph{Compared to Geologic Carbon Capture and Storage (\CCS)}
\label{product:compare_underground}

Lastly, in the realm of carbon sequestration, there currently exist
few alternatives, notably, Geologic \CCS. Our product provides a far
more sustainable method for carbon sequestration over geologic
\CCS. Geologic \CCS\ is a process in which carbon dioxide is separated
from flue gases, liquefied, then pumped underground. A key liability
in \CCS is that it would be crucial to monitor the injected sites for
several hundred years, given that a \co leak would be catastrophic,
both on a local and global level. 

\subsubsection*{IP Protection}
\label{product:ip}

Our team is in the process of developing a patent for the U.S. Patent
and Trademark Office (\textsc{uspto}). The Georgia Tech Office of
Technology and Licensing (\textsc{otl}) will perform a prior art
search and patentability assessment on the patent work we complete and
will work to finalize and submit our patent application. In addition
to the U.S. patent application, we plan on filing a placeholder with
the international community through the Patent Cooperation Treaty
(\textsc{pct}) which will allow us to file for international patents
within 30 months.

\subsection*{Other Topics}
\label{sec:othertopics}

\subsubsection*{Special Recognition:}
\label{othertopics:special}


In February 2009, the founders won second place in the “Ideas to
Serve” (\textsc{I2S}) Competition sponsored by the College of
Management at Georgia Tech for early stage venture products. The
awards from I2S competition included recognition at the Business Plan
Competition awards ceremony and a \$500 cash prize.

In March 2009 our founding entrepreneurial team won the InVenture
Prize, a competition focused on fostering innovation and
entrepreneurship at Georgia Tech. The prize for winning the
competition included a \$10,000 cash prize, free patent filing through
the Georgia Tech \textsc{OTL}, and connections with various firms
affiliated with the Georgia Tech community.

\section{Market Analysis}
\label{sec:market}

\subsection{Primary Market: Biofuel Suppliers}
\label{sec:primary-market}

Today, 12 billion gallons of biofuels are produced for domestic
consumption yearly. The majority of these biofuels are generated via
transesterification processes on corn oils; however, the governmental
goal for 2022 demands production of 36 billion gallons/year with a
hard cap at only 15 billion/year being produced from corn
feedstocks. This goal has recently been readdressed and expanded by
President Obama's \textit{Growing America’s Fuel} program with
additional organization and funding as directed by the EPA and the
USDA's Research, Economics, and Education (REE) and Forest Service
(FS) departments.

This policy is indicative of the growing movement to domestic,
sustainable, and carbon-reduced fuel sources. Societal and political
pressures are making biofuels more and more desirable and currently
many companies SHOULD I MENTION THESE COMPANIES HERE are testing lab
bench methods for biofuel creation from various biomass sources and
with a special focus on algae. This is unsurprising since of all
currently investigated biofuel feedstocks algae has a number of
distinct advantages. It is resilient to environmental changes, can be
grown very quickly under ideal nutrient conditions, is being targeted
by advanced genetics research to improve strains, and finally it is
already one of the most carbon-negative QUOTE NUMBERS HERE feedstocks
available.

With these planned expansions in biofuel production and consumption,
there is a great deal of room for the kind of large scale algae farm
our technology supports. NUMBERS NUMBERS NUMBERS?

\subsection{Other Algae Consumers}
\label{sec:other-algae-consumers}

The biofuel market may be today's most promising consumer of algae,
however algae biomass is valuable in many other markets as well. In
particular, there is interest in using algae biomass in the food,
pharmaceutical, cosmetic, and fertilizer markets. Though we do not
expect these consumers to represent a large percentage of our market,
some of their needs may develop much more quickly than the biofuels
market thus acting as support capital during Sora's development.

\subsection{Socio-Environmental Impact}
\label{sec:socio-envir-impact}

As mentioned, algae is a more carbon-negative feedstock than most
other proposed biofuel feedstocks. Already this provides a strong
societal and environmental impetus for expanding algae growth. As a
second factor, Sora's algae growth bioreactors are designed to
carefully control the algae's growth environment and plug into
nutrient sources as needed. This design allows for direct linkage of
Sora's algae farms to heavy carbon producers such as industrial
manufacturing plants or coal-fired power plants.

In particular, the US emits over 1950 million metric tons of \co\
every year and that number is expected to grow by 16.8\% over the next
20 years. Though current carbon trade markets do not value this carbon
very highly, it is very possible that new policy on carbon cap and
trade will change that. In either case, Sora's algae farms could
recycle these carbon emissions directly benefiting power plants by
disposing of excess \co\ and benefiting Sora by providing a cheap
source of needed \co.

Today over XXX coal-fired power plants exist and employ XXX AMERICANS
men and women. It is unreasonable to expect these plants to disappear
and thus unreasonable to believe the environmental damage caused will
be simply mitigated by new sources of power. Instead, Sora believes
that the algae bioreactors can help to ease this transition.

``SECOND-HAND CARBON''

\section{Competition}
\label{sec:competition}

\subsection*{Existing Competition}
\label{competition:existing}

There are many companies with technologies and business models that
pose some form of competition to us. Most generally, we perceive
businesses which produce biofuel feedstocks to be our main
competitors. These competitors are divided into two main subgroups:
companies producing ethanol and biodiesel from corn and soybeans, and
the burgeoning number of new startups centered around algae-based
biofuels.

The first type of competition to us are corn and soybean-based biofuel
producers. The market for ethanol has grown tremendously over the past
five years. This is due in large part to major government subsidies
and an ethanol mandate in the Energy Policy Act of 2005 requiring that
gasoline sold in the U.S. must contain an increasing amounts of
ethanol in coming years.

Feedstocks such as corn and soybeans have been the simplest to employ
to meet the ethanol mandate since the farmland and infrastructure to
grow these crops is readily available. However, these crops are
plagued by two main problems which put us at an advantage over
them. First, producing these fuels from corn and soybeans is energy
inefficient; indeed, more energy is required to produce the ethanol
than the energy contained in the fuel. Second, both crops require vast
tracts of land to produce sizable amounts of fuel as compared to
algae.  In particular, corn and soybeans produce 18 gallons and 50
gallons per acre per year, respectively, while algae can produce
5,000-15,000 gallons per acre per year.

The second subgroup of competitors are startup companies that are in
the algae growing business. Although there are now over twenty
startups in the algae producing field, none of them have yet moved
from the lab or pilot phase to a full-scale commercial operation. A
list of algae-related startups and their proposed markets are shown in
Table \ref{tab:competitors} in the Attachments.

The primary driver behind most of these startup companies is to
produce biofuels - ethanol and/or biodiesl - from algae. Most
companies are attempting to both grow the algae and process it into
biofuels. This business model differs from corn and soybean-based
ethanol, where the crop production is segmented from the fuel
processing.

In contrast to these algae startups, our business model focuses solely
on growing algae and does not encompass biofuel processing. In this
way, we plan to produce a feedstock of algae to sell to companies with
the technology and expertise to process it into biofuels in much the
same way that farmers across the Midwestern U.S. grow and provide corn
and soybeans to ethanol refineries.

Besides our business model, the other main difference that sets us
apart from other algae startups is our technology. The current
competitive field can be split into companies which use open pond
systems and those using closed bioreactors to grow algae. While many
companies use open ponds, those using closed bioreactor systems, such
as ourselves, (see Table \ref{tab:competitors} in the Attachments),
experience boosts in efficiency and control in excess of the
additional material and construction costs required for closed
bioreactors.

Our bioreactor has clear advantages over open systems in its ability
to maximize land efficiency and prevent external contamination. We
have a technological edge over other closed bioreactors in that our
system is designed to use cheap, natural sunlight and to maximize the
light efficiency of the algae growth process. We believe that this
design will result in lower material cost, higher land efficiency, and
more scalable installations than other closed bioreactor systems.

\subsection*{Future Competitors}
\label{competition:future}

The future competitive landscape may include other forms of renewable
energy in addition to biofuels. This will be contingent on the market
success of electric vehicles. Currently, the overwhelming market for
biofuels, such as those that will be produced from our algae, is in
transportation. If major technological advances are made in electric
vehicle technology which make them economically competitive with
traditional, hydrocarbon fuel burning vehicles, renewable energy
technologies, such as wind and solar, will be in much greater demand
to produce the electricity needed to power a growing fleet of electric
vehicles.

If electric vehicles do increase significantly in overall market
share, there will be less of a demand for biofuels for
transportation. However, if this trend does occur, we fully expect
that the algae-based biofuel market will evolve and continue to
flourish since the biofuels may shift to being used for electricity
production alongside other renewables.



\section{Marketing and Sales}
\label{sec:marketing}

\subsection*{Target Customers}
\label{marketing:target}
Our target customers are companies which own and operate coal-fired
power plants. These companies are currently under pressure to reduce
their carbon emissions. Our product provides a means by which these
companies can strategically improve their financial and legal position
by mitigating the regulatory and financial risks associated with their
carbon emissions. In comparison to other carbon sequestration
technologies, our product offers greater land efficiency and a
continuous harvest cycle.

\subsection*{Distribution System}
\label{marketing:distribution}

We plan to implement our product by making use of pre-existing plant
infrastructures and subcontractors already tailored to power plant
feed and waste processing. This cooperation is necessary due to the
capital intensive nature of our product and the need to adapt to
various scales and implementation details of each plant. Fortunately,
our solution can fit this niche well as many components are already
popular in industrial processing and novel components can be easily
constructed.

\begin{comment}
\subsection*{Pricing Policy}
\label{marketing:pricing}
As we currently only have a very rough estimate of the cost needed to
build our system, we believe that at the early commercial stage, we
would price each bioreactor to make a 10 to 12\% profit over the total
cost required to make the bioreactor. This above sentence doesn't
actually describe a cause-and-effect relationship, or its a really
weird one that doesn't make much sense to me.
\end{comment}

\subsection*{Marketing \& Sales Plan}
\label{marketing:plan}
We plan on selling our product by first developing a strong working
relationship with our customers, initiated by touring their facilities
and exhibiting to them our proof of concept. We then plan on working
directly with our customers on a case-by-case basis to establish how
they can best adopt our product to their specific power plant. This
would include the creation of an incremental installation plan. Since
our product requires a significant capital investment and is a major
shift from our customers' current operations, a strong contact with
our customer base will be critical to the successful adoption of our
product. We have already begun to initiate these contacts with various
power producers in the Southeast.

\section{Operations and Manufacturing}
\label{operations}

\begin{comment}
\subsection*{Operations and Manufacturing Strategy}
\label{operations:strategy}
The production of the full algae farm system will be
distributed between the various subsystems. The primary subsystems are
growth regions, nutrient mixing chambers, water coolers, \co\
diffusion systems, and algae harvesting and drying systems. The growth
regions, comprised of a network of clear polymer tubes, will be
provided by an external manufacturer and cut to our specifications.

\end{comment}

\subsection*{The Operations and Manufacturing Process}
\label{operations:process}
A firm versed in large-scale construction projects and
familiar with projects involving the transport of fluids will be
responsible for supervision of the construction. This company's
responsibilities will include facilitation of the construction process
and oversight of the integration of the various subsystems at each location.

The specifics of the manufacturing process require detailed discussion of
the technology. These details are given in a patent application currently
being processed by the U.S. Patent and Trademark Office and thus will not presently
be divulged.

\begin{comment}
  The bioreactor growth tubes can be made of a material transparent to
  visible sunlight, such as silicone, vinyl, or clear PVC. The growth
  tubes will be purchased wholesale from a major distributor and
  delivered to the bioreactor site. On site, the tubes will first be
  braided together into long strands.  The number of tubes in each
  strand, and the length of each strand, will be dictated by
  geographic conditions at the bioreactor site. For example, if the
  average solar flux intensity at the site is relatively high, more
  tubes will be braided together in order to distribute the light to
  the algae most effectively. Alternatively, if the solar flux is
  relatively low, fewer tubes will be braided together in order to
  maximize the amount of light within each tube. The length of each
  strand will be determined based on the geography and topography of
  the bioreactor site. Following the tube braiding, the strands will
  be woven into a 2-dimensional matrix. The particular variety of
  weaving pattern will again be dependent on geographic conditions at
  the bioreactor site, in particular the average solar flux
  intensity. The size of the matrix will be dependent on the land
  available at the site.

  Nutrient mixing will be accomplished in a mixing chamber located
  adjacent to one or more areas of growth tubes, depending on the size
  of the bioreactor site. These mixing chambers will contain the
  nutrients to be supplied to the algae in concentrated form.  Each
  nutrient mixing chamber will be fitted with automated hopper systems
  which control the rate of nutrient injection into the growth tubes.
  Additionally, a mixer system will be in place to increase the rate
  at which the nutrients dissolve. The components necessary for
  nutrient mixing, namely the hoppers and the driving mechanism, will
  be obtained through suppliers in those industries.

  The water cooling system will utilize any one of a number of
  different types of industrial scale water chillers commercially
  available from wholesale manufacturers. Similar to the nutrient
  mixing chamber, the water cooling system will be located in a
  centralized location with direct access to one or more adjacent
  sections of growth tubes. The water from each of section of growth
  tubes will be piped into the cooling system to extract the heat
  generated by solar infrared absorption and from excess heat
  introduced by the \co\ gases.

  The algae harvesting and drying system will be installed in a
  central location with respect to several sections of woven growth
  tubes. This system will consist of several components, each of which
  will be purchased from a wholesale distributor. First, an algae
  collection chamber will be constructed to hold a large quantity of
  algae-laden water. The algae collection chamber may be outfitted
  with nozzles to inject a commercially available flocculant, such as
  Prestol, into the algae solution to aggregate and concentrate the
  algae solution in the bottom of the chamber. Alternatively, the
  chamber may be fitted with an outlet through a ceramic micro-flow
  filter to concentrate the algae solution into a separate, smaller
  chamber. Downstream from the chamber any one of several commercially
  available drum presses will be installed to dry the concentrated
  algae solution for shipment. As with the other subsystems, the
  integration will be supervised by the managing construction firm.

  Though the piping for the liquids and the ductwork for the gases
  will be sourced separately from independent suppliers, integration
  of the overall system will be completed by the managing construction
  firm after the materials are delivered to the site. The pumps and
  the forced draft fan, used to facilitate the transfer of fluids,
  will be obtained and incorporated in a similar manner.
\end{comment}

\subsection*{Key Suppliers and Subcontractors}
\label{operations:suppliers}
One of the key processes involved in the creation of the sequestration
system is the integration of the various subsystems. This process will
require the services of a construction firm familiar with the concerns
involved in construction of emission control systems at coal-fired
power plants. This firm will be chosen for its combination of
expertise in engineering, construction, fabrication, and project
management. Additionally, this firm should have an existing customer
base within the power industry and have worked on emissions mitigation
systems with power providers in the past. Potential firms under future
review include Alberici Enterprises, Lauren Engineers and
Constructors, Kiewit Energy, and Bechtel Power.

Certain components of the carbon sequestration system, such as the
hopper, mixer, piping, ductwork, fans, and pumps will be obtained from
the appropriate commercial suppliers. The choice of suppliers will
ultimately be determined by the location of the power plant site on a
case-by-case basis, allowing for minimization of transportation costs
and other potential cost-related benefits to working with the local
supply chain.

\subsection*{Financial Characteristics}
\label{operations:financial}
Sora's financial outlook in its first year (June 2010 - June 2011) can
be separated into three main types of expenses: 1) pre-pilot
construction costs, 2) pre-pilot operating costs, and 3) business
operation expenses. We aim to begin construction of a pre-pilot system
of our algae-based sequestration system within the coming year. The
pre-pilot will be our first major step on the road to a full-size,
commercial scale sequestration system.  Its purpose will be to prove
the concept of our technology, to optimize the energy and space
efficiency of the technology, and to attract potential investors and
customers for a pilot operation. The projected cost for the materials
to construct the pre-pilot are given in Tables \ref{tab:costs1} and
\ref{tab:costs2} in the Attachments, and totals approximately
\$30,000.

The second main expense type for Sora's first year will be the
operating costs for the pre-pilot.  These expenses include utilities,
such as power and water, along with necessary nutrients to support the
operation of the pre-pilot. The projected cost for these items is
given in Table \ref{tab:costs3} in the Attachments and totals
approximately \$30,000.

The final expense type is Sora's business-related expenses. These
expenses include office needs, marketing, insurance, legal fees, and
intellectual property protection.  The projected cost for each of
these is given in Table \ref{tab:costs4} in the Attachments and totals
approximately \$450,000.

\section{Management}
\label{sec:management}

\subsection*{Founding Entrepreneurial Team}
\label{management:founders}

\subsubsection*{Will Boyd, \textit{President}}

Will is a 4th year Physics and Computer Science major at the Georgia
Institute of Technology. Will's interest in founding Sora comes from
his interest in carbon mitigation and renewable energy technologies
and business strategies to employ them in the marketplace. In 2007,
Will co-founded Trailblazers, a student organization devoted to
promoting the values of environmental conservation, and has actively
been involved in building the organization since its inception. Will
has been actively involved in undergraduate research in physics and
computing both at Georgia Tech and during the summer of 2009 at the
\textsc{atlas} experiment at the European Organization for Nuclear
Research (\textsc{cern}) in Geneva, Switzerland.

\subsubsection*{Andrew Punnoose, \textit{Vice President of Research \& Development}}

Andrew is a 4th year Aerospace Engineering student at the Georgia
Institute of Technology. He is currently an undergraduate researcher
in the Unmanned Aerial Vehicle (\textsc{uav}) Lab, working on
incorporation of optic flow sensing into helicopter navigation
technology.  Andrew's interest in founding Sora was a result of his
fascination with biological systems and his interest in renewable
energy. Andrew plans on working and making advancements in the field
of robotics, and hopes to contribute to the exploration of space, the
final frontier.

\subsubsection*{Sanjay Challa, \textit{Vice President of Business Development}}

Sanjay is a 4th year Biomedical Engineering major at Georgia
Tech. Sanjay worked at National Instruments (NI) as an Engineering
Leadership Program Intern, where he built a medical device from
scratch on the technical marketing team to showcase some of NI's
technology, and set an industry standard in \textsc{iso} and
\textsc{fda} compliance documentation for the use of NI's products in
creating medical devices. Sanjay is also the Curriculum Chair on the
Biomedical Engineering Student Advisory Board, where he serves as a
liaison between students and faculty to drive decisions on curriculum
and class changes within his major. Lastly, Sanjay has taken
design-based classes focused on teaching students the entire product
design process, emphasizing market analysis, manufacturing methods and
materials, and various design techniques.

\subsubsection*{Joseph Abrahamson, \textit{Vice President of Marketing \& Sales}}

Joseph is a 4th year Biomedical Engineering major at Georgia
Tech. Joseph has made advanced studies in mathematics and chemistry
while practicing poignant graphic design and working in
neuroengineering labs in his downtime. Joe has always been interested
in developing alternative energies and so eagerly took up the chance
to start Sora.

\subsubsection*{Kento Masuyama, \textit{Vice President of Finance}}

Kento is a 4th year Aerospace Engineering student at the Georgia
Institute of Technology. Kento currently serves as secretary of the GT
Sport Parachute Club. He was a member of the 1st place team in the
2007 \textsc{aiaa} Undergraduate Team Space Transportation System
Competition, and he currently works in the Space Systems Design
Laboratory at Georgia Tech. His interest in this project stems from a
fascination of and love for nature, as well as the prospect of
overcoming an important challenge. Kento hopes to go to graduate
school in aerospace engineering, and eventually work in research and
development for space exploration.

\subsection*{Future Management Roles}
\label{management:future}

In the future, we plan to add additional positions to our management
and technical teams as they become necessary. In particular, it will
be advantageous for us to hire a “Go-to-Market” \textsc{ceo} as we
progress from our pre-pilot to pilot phase. For this position, we plan
to recruit an individual with extensive experience in the chemical
processing industry and in planning and building pilot and/or demo
stage plants. In addition, we expect to add a \textsc{cfo} and
\textsc{cto} as we transition from the pilot phase to demo phase.  We
plan to fill both of these roles with individuals with experience
building technology-oriented start-up companies, though the
\textsc{cto} role may be filled by one of our founding members.

\subsection*{Board of Advisors}
\label{management:advisors}

\subsubsection*{L. Franklin Bost, \textit{\footnotesize MBA, ISDA}}

Mr. Bost is currently Director of Design Instruction at the Wallace
H. Coulter Department of Biomedical Engineering, Georgia Institute of
Technology and Emory University. He teaches sophomore design and the
capstone senior design classes.

Mr. Bost is President and Chief Executive Officer of SpherIngenics,
Inc. and early-stage biotechnology company specializing in stem cell
delivery technologies. He holds an \textsc{mba} from the University of
North Carolina and a Bachelor of Product Design degree from North
Carolina State University. He is a member of the Industrial Designers
Society of America (\textsc{idsa}), Society of Biomaterials,
Biomedical Engineering Society, and the American Association for the
Advancement of Science. From 1994 to 2006, Mr. Bost serviced on
AdvaMed's Technology and Regulatory committee.

Mr. Bost has over 35 years of business experience in strategic
planning, new product development, senior management and leadership
with companies in medical, consumer and industrial markets. He was
previously president of Porex Surgical Inc. a biomaterials company,
which developed, manufactured and marketed implantable products for
craniofacial reconstruction. Porex Surgical grew to service surgeons
in five medical sub-specialties in the U.S. and over 50 international
markets including craniofacial, plastic and reconstructive,
oculoplastic, otolaryngologist, oral maxillofacial and
neurosurgeons. Previously, he held senior management positions with
Porex Corporation, a medical, consumer and industrial products
company, for U.S. and international marketing and sales
development. Mr. Bost has worked for American Hospital Supply in
product development and with Becton Dickinson in their hemodialysis
business.

\subsubsection*{Ben Hill, \textit{\footnotesize MBA}}

Ben Hill is currently Senior Business Advisor of VentureLab, a
product development and commercialization group for early stage
ventures at Georgia Tech. As a member of VentureLab, Ben focuses on
clean technology research and commercialization. In addition, Ben is
an adjunct professor in the Georgia Tech College of Management. Prior
to joining VentureLab, Ben was the co-founder of Cirronet, Inc., a
wireless products company that was later acquired by RF
Monolithics. Ben has been an invaluable member of the Board of
Advisors since Sora's inception. Ben holds an \textsc{mba} from Emory
University and a Master's of Arts in Religion from Yale University.
 
\section{Future Growth}
\label{sec:growth}

\subsection*{Positive Contingencies}
\label{growth:positive_contigencies}
There are two major contingencies that could provide a major boost to
our market for carbon sequestration in the near future. Current
climate legislation moving through Congress, such as the Waxman-Markey
bill and the Kerry-Boxer bill, would be a major positive contingency
for our business model, as it would attach a significant cost to
carbon dioxide emissions. Such a cost would drive a change in the
operations of a variety of industries through the basic principles of
economics. Climate legislation will most likely come in the form of a
cap and trade system. This would be a major asset to our business as
carbon emissions do not presently pose a financial burden to emitters
(although such industries can already reduce their emissions and sell
carbon credits to help recoup the costs incurred).
 
A second major positive contingency would be the passage and/or
renewal of substantial subsidies for biofuel production. Indeed,
recent U.S. legislation requires vast increases each year in ethanol
production and its use as an additive in gasoline. This legislation
has established ethanol as a permanent commodity and instilled
confidence in biofuel startup ventures. Our business model places us
in a position to benefit from these and other federal, state, or local
policies that benefit biofuel producers. Government incentives for
biofuels would in effect increase the market potential for algae as
feedstock for biofuels, making our product more profitable for
coal-burning power plants.

\subsection*{Areas for Future Growth \& Development}
\label{growth:areas}
There are a number of avenues for growth that we plan to explore as
Sora develops in the next five years. The first of these is
international intellectual property protection. At present, the
U.S. is behind the curve on clean energy and climate legislation
compared to many other developed nations, including those in the EU
and Japan. As a result of this discrepancy, carbon credits are
currently worth significantly more in those countries than in the
United States. As such, it could prove quite beneficial to our
business to file patents in one or more of these countries to have
access to more lucrative carbon markets overseas. A second area for
further growth is other \co\ producing industries, including natural
gas and petroleum-burning power plants, oil refineries, paper mills,
and concrete factories.

\section{Attachments}
\label{sec:attachments}

\newpage

\begin{sidewaystable}[h!]
  \footnotesize
  \centering
  \begin{tabularx}{,85\textwidth}{llXXX}
    \toprule
    \textbf{Company}                   & \textbf{Proposed Market}                                    & \textbf{Status}  & \textbf{Website}         \\
    \hline
    Algenol Biofuels                   & Ethanol production, \co sequestration                       & Pilot phase      & algenolbiofuels.com  \\
    Aquaflow Bionomic Corp             & Wastewater treatment, Biomass production                    & Pilot phase      & aquaflowgroup.com    \\
    Aquatic Energy                     & Biomass production, \co sequestration                       & Pilot phase      & aquaticenergy.com    \\
    Aurora Biofuels                    & Biodiesel production                                        & Pilot phase      & aurorabiofuels.com   \\
    Bionavitas                         & Biodiesel production, \co sequestration                     & Lab phase        & bionavitas.com       \\
    Blue Marble Energy                 & Wastewater treatment, Biomass/natural gas production        & Pilot phase      & bluemarble energy    \\
    Bodega Algae                       & Wastewater treatment, Biomass production, \co sequestration & Lab phase        & bodegaalgae.com      \\
    Carbon Capture Corp                & Biomass production, \co sequestration                       & Pilot phase      & carbcc.com           \\
    Cellana                            & Biomass production, Biofuel processing                      & Pilot phase      & cellana.com          \\
    GreenFuel Technologies Corp        & Biodiesel/ethanol production, \co sequestration             & \textit{Defunct} & greenfuelsonline.com \\
    INFINIFUEL Biodiesel               & Biodiesel production                                        & Lab phase        & infinifuel.com       \\
    Inventure Chemical Technology      & Biodiesel/ethanol production                                & Pilot phase      & inventurechem.com    \\
    Kent BioEnergy Corp                & Wastewater treatment, Biomass production, \co sequestration & Pilot phase      & kentbioenergy.com    \\
    LiveFuels                          & Biocrude production                                         & Lab phase        & livefuels.com        \\
    OriginOil Incorporated             & Biomass production, Biomass processing                      & Commercial phase & originoil.com        \\
    PetroAlgae                         & Carbon sequestration                                        & Pilot phase      & petroalgae.com       \\
    PetroSun Incorporated              & Biodiesel production                                        & Pilot phase      & petrosuninc.com      \\
    Sapphire Energy                    & Biocrude production                                         & Pilot phase      & sapphireenergy.com   \\
    Seambiotic                         & Biomass production, \co sequestration                       & Pilot phase      & seambiotic.com       \\
    Solena Group                       & Syngas production                                           & Pilot phase      & solenagroup.com      \\
    Solix Biofuels                     & Biodiesel production, \co sequestration                     & Pilot phase      & solixbiofuels.com    \\
    XL Renewables                      & Biomass production, Wastewater treatment, \co sequestration & Pilot phase      & xldairygroup.com     \\
    \bottomrule
  \end{tabularx}
  \caption{An overview of current indirect competitors.}
  \label{tab:competitors}
\end{sidewaystable}

 \begin{sidewaystable}
  \footnotesize
  \centering
  \begin{tabular*}{\textwidth}{p{5cm} p{4cm} p{5cm} p{2cm} p{2cm} p{2cm}}
    \toprule
    \textbf{Item}     & \textbf{Supplier}    & \textbf{Note}     & \textbf{Qty}    & \textbf{Cost/Item}    & \textbf{Total Cost} \\
    \hline
    \\
    \multicolumn{6}{c}{\textbf{Bioreactor}}\\ 
    \hline
    \\
    Silicone tubing  & Ryan Herco Flow Solutions & Support pre-pilot & 300 ft & \$1.2/ft & \$600 \\
    Low shear water pumps   & ? & ? & ? & ? & ? \\
    \\
    \multicolumn{6}{c}{\textbf{CO$_2$ Dissolution System}}\\ 
    \hline
    \\
    Bubble breaking mesh    & Woven Wire    & Support pre-pilot    & 1    & \$60    & \$60 \\
    CO$_2$ hose    & Marine Depot    & Support pre-pilot    & 100 ft.    & \$0.99/ft    & \$100 \\
    CO$_2$ regulator    & Marine Depot    & Support pre-pilot    & 1    & \$150    & \$150 \\
    CO$_2$ tanks    & Cylinder Mart    & Support pre-pilot    & 2    & \$115    & \$230 \\
    Dissolution tank    & ? & ? & ? & ? & ? \\
   Water pump    & Grainger    & Support pre-pilot    & 4    & \$106    & \$424 \\
    \\
    \multicolumn{6}{c}{\textbf{Control System}}\\ 
    \hline
    \\
    Backup hard drive    & Western digital    & Support pre-pilot    & 1    & \$250    & \$250 \\
    Cables    & Various    & Support pre-pilot    & N/A    & \$500    & \$500 \\
    Computer    &   Various    & To support pre-pilot    & 1    & \$500    & \$500 \\
    DAQ BNQ    & National Instruments    & Support pre-pilot    & 1    & \$2,000 & \$2,000 \\
    DAQ board    & National Instruments    & Support pre-pilot    & 1    & \$800    & \$800 \\
    Flow sensors    & RCM Industries    & Support pre-pilot    & 5    & \$120    & \$600 \\
    Light sensors    & Macam Photometrics Ltd.    & Support pre-pilot    & 4    & \$30    & \$120 \\
    Power supply    & Sola HD    & Support pre-pilot    & 1    & \$1,000    & \$1,000 \\
    \\
    \multicolumn{6}{c}{\textbf{Lab Supplies}}\\ 
    \hline
    \\
    Mass Spec. (time-shared)    & ?    & ?     & ?     & ? & ? \\
    Balance    & Mettler Toledo    & Algae productivity analysis       & 1    & \$2,200    & \$2,200 \\
    Cell counter    & Beckman Coulter    & Algae productivity analysis     & 1    & \$10,000    & \$10,000 \\
    Cuvette Rack    & International Crystal    & Store algae samples               & 1    & \$45     & \$45 \\
    Glassware    & Indigo Instruments    & Beakers, burets, pipets        & 1    & \$200    & \$200 \\
    Lab tables    & Hertz Furniture Systems    & Support lab equipment    & 2    & \$225    & \$450 \\
    Optical Cuvettes    & International Crystal    & Transport algae samples         & 1    & \$850    & \$850 \\
    Optical microscope    & ? & ? & ? & ? & ? \\
    UV-vis Spectrophotometer    & Hewlett-Packard    & Characterize algae samples    & 1     & \$8,000    & \$8,000 \\
    \bottomrule
  \end{tabular*}
    \caption{Pre-Pilot Construction Costs}
    \label{tab:costs1}
    \end{sidewaystable}

\begin{sidewaystable}
  \footnotesize
  \centering
  \begin{tabular*}{\textwidth}{p{5cm} p{4cm} p{5cm} p{2cm} p{2cm} p{2cm}}
    \toprule
    \textbf{Item}     & \textbf{Supplier}    & \textbf{Note}     & \textbf{Qty}    & \textbf{Cost/Item}    & \textbf{Total Cost} \\
    \hline
    \\
    \multicolumn{6}{c}{\textbf{Nutrient Mixing System}}\\ 
    \hline
    \\
    Automated valve    & Clippard Instrument Laboratory  & Support pre-pilot    & 2    & \$50    & \$100 \\
    Dry chemical mixing tank & ?    & Support pre-pilot    & ? & ? & ? \\
    Water pump    & Grainger    & Support pre-pilot    & 2    & \$106    & \$212 \\
    Water filter    & ? & ? & ? & ? & ? \\
    \\
    \multicolumn{6}{c}{\textbf{Shipping Fees}}\\ 
    \hline
    \\
    \\
    \multicolumn{6}{c}{\textbf{Tools}}\\ 
    \hline
    \\
    Circular saw    & Amazon    & Construct/maintain pre-pilot    & 1    & \$160    & \$160 \\
    Clamps    & Amazon    & Construct/maintain pre-pilot    & N/A    & \$105    & \$105 \\
    Cordless drill    & Amazon    & Construct/maintain pre-pilot    & 1    & \$350    & \$350 \\
    Drill bits    & Amazon    & Construct/maintain pre-pilot    & N/A    & \$150    & \$150 \\
    Electric sander    & Amazon    & Construct/maintain pre-pilot    & 1    & \$70    & \$70 \\
    Grinder    & Amazon    & Construct/maintain pre-pilot    & 1    & \$90    & \$90 \\
    Hammers    & Amazon    & Construct/maintain pre-pilot    & N/A   & \$100    & \$100 \\
    Outdoor lighting    & Amazon    & Construct/maintain pre-pilot    & 2    & \$80    & \$160 \\
    Pipe wrenches    & Amazon    & Construct/maintain pre-pilot    & 1     & \$15    & \$15 \\
    Pliers    & Amazon    & Construct/maintain pre-pilot     & N/A    & \$35    & \$35 \\
    Power cords    & Amazon    & Construct/maintain pre-pilot    & 10    & \$35    & \$350 \\ 
    Sockets    & Amazon    & Construct/maintain pre-pilot    & N/A    & \$60    & \$60 \\
    Soldering iron    & Amazon    & Construct/maintain pre-pilot    & 1    & \$45    & \$45 \\
    Wrenches    & Amazon    & Construct/maintain pre-pilot    & N/A    & \$65    & \$65 \\
    \\
    \multicolumn{6}{c}{\textbf{Water Cooling System}}\\ 
    \hline
    \\
    Water chiller & Aqua Logic & Support pre-pilot & 3 & \$1,500 & \$4,500 \\
    \\
    \multicolumn{6}{c}{\textbf{Total Projected Pre-Pilot Construction Cost}}\\ 
    \hline
    \\
    \textbf{Total Cost} & & & & & \textbf{\$30,632} \\
    \bottomrule
    \end{tabular*}
    \caption{Pre-Pilot Construction Costs (cont.)}
    \label{tab:costs2}
    \end{sidewaystable}

\begin{sidewaystable}
  \footnotesize
  \centering
  \begin{tabular*}{\textwidth}{p{5cm} p{4cm} p{5cm} p{2cm} p{2cm} p{2cm}}
    \toprule
    \textbf{Item}     & \textbf{Supplier}    & \textbf{Note}     & \textbf{Qty}    & \textbf{Cost/Item}    & \textbf{Total Cost} \\
    \hline
    \\
    \multicolumn{6}{c}{\textbf{Nutrients}}\\ 
    \hline
    \\
    KNO$_3$ & Alfa Aesar   & Potassium Nitrate & 22.5kg/day    & \$30   & \$10,950\\
    KH$_2$PO$_4$    & Alfa Aesar    & Potassium dihydrogen phosphate    & 560kg/day & \$21   & \$7,665 \\
    MgSO$_4$ (7$\times$H$_2$O)   & Alfa Aesar    & Magnesium sulfate & 1,125kg/day   & \$26/day    & \$9,490 \\
    FeSO$_4$ (7$\times$H$_2$O)   & Alfa Aesar    & Iron(II) sulfate heptahydrate & 1.35kg/day    & \$0.06/day & \$22 \\
    H$_3$BO$_3$    & Alfa Aesar    & Boric acid    & 1.25kg/day & \$0.03/day & \$11 \\
    MnSO$_4$ (7$\times$H$_2$O)   & Alfa Aesar    & Manganese sulfate    & 1.2kg/day & \$0.03/day & \$11 \\
    ZnSO$_4$ (7$\times$H$_2$O)   & Alfa Aesar    & Zinc sulfate  & 0.1kg/day & \$0.004/day    & \$1.45 \\
    CuSO$_4$ (5$\times$H$_2$O)   & Alfa Aesar    & Copper sulfate    & 36mg/day  & \$0.0007/day   & \$0.25 \\
    Na$_2$MoO$_4$   & Alfa Aesar    & Sodium molybdate  & 10mg/day  & \$0.0008/day   & \$0.30 \\
    CO$_2$  & Pye-Barker, Fire and Safety  & Carbon dioxide & 10lb/month & \$0.85/day & \$305 \\
    Dechlorinator & ? & ? & ? & ? & ? \\
    Water filter cartridges & ? & ? & ? & ? & ? \\
    \\
    \multicolumn{6}{c}{\textbf{Utilities}}\\ 
    \hline
    \\
    Power & Georgia Power & Run pre-pilot & 42,000 kWh & \$6.33/day & \$2,310 \\
    Water & City of Atlanta & Support pre-pilot & 120,000 gal & \$6.67/day & \$2,400 \\    
    \\
    \multicolumn{6}{c}{\textbf{Total Projected Pre-Pilot Operating Cost}}\\ 
    \hline
    \\
    \textbf{Total Cost} & & & & & \textbf{\$33,166} \\
    \bottomrule
    \end{tabular*}
    \caption{Pre-Pilot Operating Expenses (June 2010 - June 2011)}
    \label{tab:costs3}
    \end{sidewaystable}


\begin{sidewaystable}
  \footnotesize
  \centering
  \begin{tabular*}{\textwidth}{p{5cm} p{7cm} p{5cm} p{2cm} p{2cm} p{2cm}}
    \toprule
    \textbf{Item}   & \textbf{Note}     & \textbf{Qty}    & \textbf{Cost/Item}    & \textbf{Total Cost} \\
    \hline
    \\ 
    \multicolumn{5}{c}{\textbf{Labor}}\\ 
    \hline
    \\
    Employee salaries   & Atlanta cost of living adjusted    & 5 founders $\times$ 12 months    & \$2,000  & \$120,000 \\
    \\
    \multicolumn{5}{c}{\textbf{Conferences and Networking}}\\ 
    \hline
    \\
    Algal Biomass Summit & Registration, flights, etc. & 5 attendees & \$2,350 & \$11,750 \\
    Carbon Capture and Sequestration Summit & Registration, flights, etc. & 5 attendees & \$1,300 & \$6,500 \\
    Electric Power Conference & Registration, flights, etc. & 5 attendees & \$1,400 & \$7,000 \\
    National Algae Association Conference & Registration, flights, etc. & 5 attendees & \$1,300 & \$6,500 \\
    Power-Gen Conference & Exhibit, registration, flights, etc. & 5 attendees & \$2,600 & \$13,000 \\
    \\
    \multicolumn{5}{c}{\textbf{Insurance}}\\ 
    \hline
    \\
    Business owners policy  & ? & ? & Quote req. & Quote req. \\
    Workers compensation    & ? & ? & Quote req. & Quote req. \\
    \\
    \multicolumn{5}{c}{\textbf{Legal Expenses}}\\ 
    \hline
    \\
    Attorney fees & Contractual agreements, financing, etc & 30 hours & \$500/hour & \$15,000 \\
    Patent filings  & Australia, Brazil, Canada, China, Germany, India, Japan, Russia, South Africa, Spain, U.K.     & 11 countries    & \$25,000/country & \$275,000 \\
    \\
    \multicolumn{5}{c}{\textbf{Office Supplies}}\\ 
    \hline
    \\
    Business cards  &   Marketing/networking   & 5 founders $\times$ 200   & ? & ? \\
    Computers & ? 2 & ? & ? & ? \\
    Furnishings & ? & ? & ? & ? & ? \\
    Internet & ? & ? & ? & ? & ? \\
    Office space    &   Rent and utilities    & 1   & ? & ? \\
    Phone line & ? & ? & ? & ? & ? \\
    Printer paper   &   General purpose printing    & 10 reams  & ? & ? \\
    Printer/scanner/fax machine &   General purpose use    & 1 & ? & ? \\
    Quicken & ? & ? & ? & ? & ? \\
    SolidWorks & ? & ? & ? & ? & ? \\
    Web domain  &   Marketing   & 1 year    & ? & ? \\
    \\
    \multicolumn{5}{c}{\textbf{Total Projected Cost}}\\
    \\
    \hline 
    \textbf{Total Cost} & & & & \textbf{\$444,050} \\
    \bottomrule
    \end{tabular*}
    \caption{Business-Related Expenses (June 2010 - June 2011)}
    \label{tab:costs4}
    \end{sidewaystable}

\end{document}
